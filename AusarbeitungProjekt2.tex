\documentclass[11pt,a4paper]{article}
\usepackage{geometry}
\geometry{a4paper,left=25mm,right=25mm, top=24mm, bottom=24mm}

\usepackage{amsmath,amssymb,amstext,amsthm}
\usepackage{optidef}
%\usepackage{ucs}
\usepackage[utf8x]{inputenc}
%\usepackage[T1]{fontenc}
\usepackage[ngerman]{babel}
%\usepackage[ngerman]{varioref}
\usepackage{xcolor}
\usepackage{babelbib}
\usepackage[hidelinks]{hyperref}
\usepackage{tikz}
\usepackage{tikz-cd}
\usepackage{caption}
\usepackage[linesnumbered]{algorithm2e}



\DeclareMathOperator*{\Max}{max}
\newcommand{\N}{\mathbb{N}}
\newcommand{\Z}{\mathbb{Z}}
\newcommand{\Q}{\mathbb{Q}}
\newcommand{\TODO}{\textcolor{red}{TODO}}

\newtheoremstyle{my_th_style1}%
{6pt}%		// space above
{0pt}%		// space below
{}%		// body font
{}%		// indent amount
{\bfseries}%	// theorem head font
{:             }%		// punctuation after theorem head
{.5em}%	// space after theorem head
{}%		// theorem head spec

\theoremstyle{my_th_style1}
\newtheorem{satz}{Satz}

\makeatletter 
\renewenvironment{proof}[1][\proofname]{\par 
	\pushQED{\qed}% 
	\normalfont \topsep6\p@\@plus6\p@\relax 
	\trivlist 
	\item[\hskip\labelsep 
	%         \itshape 
	\bfseries 
	#1\@addpunct{:}]\ignorespaces 
}{% 
\popQED\endtrivlist\@endpefalse 
} 
\makeatother 

%opening
\title{Projekt 2: Fiber-To-The-x}
\author{Moritz Hefner und Lucia Ortjohann}

\begin{document}
\maketitle
\thispagestyle{empty}
\newpage
\tableofcontents
\thispagestyle{empty}
\newpage
\setcounter{page}{1}

%\section{PROBLEME}
% \begin{itemize}
%	\item DISKUNKTE VEREINGIGUNG VON WEGEN GENAUER ERKLÄREN
% 	\item ALLLES SSOOOOO HÄÄSSLICH
% 	\item P2PG was tun wir genau?
% 	\item Problemstellung auch die PRObleme genau beschreiben oder nur die %Daten und wann beschreibe ich welches Problem nochmal
 %	\item Name für Facility
 %	\item constraint in das modell/die modelle einfügen für assEdges = 0 falls endknoten nicht mit customer uebereinstimmt
% \end{itemize}


\section{Problemstellung}

Dieses Projekt behandelt die Problemstellung in einem Netzwerk von Kunden und Verbindungswegen diese Kunden mit einer Internetverbindung aus einer Leitstelle zu versorgen.
Dabei k\"onnen die Kunden entweder via Glasfaser- oder \"uber Kupferkabel mit jeweils verschiedenen Kosten und verschiedenen Bandbreiten angeschlossen werden.
Jeder Kunde besitzt eine individuelle Nachfrage an Bandbreite, die gedeckt werden muss.
Gesucht ist ein kosteng\"unstigster Standort f\"ur eine Leitstelle und eine Verbindung von dieser zu jedem Kunden.
Dabei unterscheiden wir zwischen zwei verschiedenen Problemstellungen.
Beim Point-to-point (P2P) Problem soll jeder Kunde eine eigene Glasfaserleitung bis zur letzten Anschlusskante (letzte Meile) bekommen.
Dahingegen dürfen beim Point-to-multipoint (P2MP) Problem sogenannte Splitter installiert werden, die mehrere Glasfaserkabel zusammenfassen k\"onnen.
 
Wir formulieren und lösen diese Probleme auf einem Graphen \(G=(V,A)\).
Dabei besteht die Knotenmenge aus vier disjunkten Mengen $L,K,F,S \subseteq V$, wobei $L$ die Menge der m\"oglichen Leitstellen ist. Weiterhin ist $K$ die Menge der zu versorgenden Kunden und $F$ ist eine Menge von so genannten Facility Knoten.
Von einer Facility aus können Kunden mit Glasfaser oder Kupfer angeschlossen werden.
Die Menge $F_1 \subseteq F$ ist die Menge der Facility Knoten von Typ 1.
Von diesen aus können Kunden ausschließlich über Glasfaserverbindungen angeschlossen werden.
Die Facility Knoten, von denen Kunden mit Kupfer angeschlossen werden können, bilden die Menge $F_2$. Dabei sind die Mengen $F_1$ und $F_2$ disjunkt.
Ferner bildet $S$ die Menge der Knoten, die für die Verbindung der Facility Knoten zu einer Leitstelle genutzt werden k\"onnen (Steinerknoten). 

Die Menge der Kanten $A$ besteht aus den inneren Kanten $I$ und den Anschlusskanten, wobei unterschieden wird, ob diese Verbindung mit Kupfer $A_2$ oder Glasfaser $A_1$ möglich ist.
Hier gilt, dass $A_1$ und $A_2$ disjunkt sind.
Eine Anschlusskante geht immer von einem Knoten der Menge $F$ zu einem Kunden $k \in K$. 
Im Folgenden betrachten wir $G$ als gerichteten Graphen.
Dazu fassen wir die Anschlusskanten als Kanten von der Facility zum Kunden auf.
Für jede innere Kante $ij \in I$, außer für die Kanten mit $i \in L$, fügen wir zusätzlich noch die Kante $ji$ zu $I$ hinzu.
Dieser Graph bildet das Grundgerüst für unsere Probleme.
Zusätzlich gibt es noch einige Bedingungen.
Grundsätzlich wollen wir die Kosten des Netzwerks minimieren. 
Dazu ist die Kostenfunktion $c: V \cup A \rightarrow \Q_{\geq 0}$ gegeben.
Für die Verlegung von Glasfaser auf den inneren Kanten $I$ fallen Kosten in Höhe von $c(e)$ für alle $e \in I$ an. 
Für die Verlegung von Kupfer auf den Anschlusskanten von Typ 2 fallen ebenfalls Kosten in Höhe von $c(e)$ für alle $e \in A_2$ an. 
Die  Anschlusskanten vom Typ 1 haben die Länge 0 und somit keine Kosten.
Der Aufbau einer Leitstelle $ l\in L$ kostet $c(l)$. 
In den Facility Knoten kann man entweder einen DSL-Zugangsmultiplexer (DSLAM) installieren (Facility Knoten von Typ 2) oder einen Kunden an Glasfaser anschließen (Facility Knoten von Typ 1).
Dabei ist zu beachten, dass ein Kunde lediglich dann \"uber ein Kupferkabel von einem Facility Knoten angeschlossen werden kann, wenn in diesem ein DSLAM installiert ist.
Dafür entstehen Kosten in Höhe von $c(f)$ für alle $f \in F$.
Zusätzlich hat jeder Kunde \(k \in K\) noch eine Nachfrage an Bandbreite von $d(k)$ und für den Anschluss eines Kunden kann mit Profit von $p_1(k)$ für Anschluss mit Glasfaser und $p_2(k)$ für den Anschluss mit Kupfer gerechnet werden.
Außerdem enstehen bei P2MP noch Kosten für die Installation eines Splitters in H\"ohe von $c_s$.

Die Ergebnisse der einzelnen Probleme auf drei vorgegebenen Instanzen werden jeweils am Ende des Abschnittes beschrieben.
Um die Laufzeiten der Algorithmen besser vergleichen zu können, geben wir nun die technischen Daten der genutzten Computer an:
\begin{table}[h]
	\centering
	\begin{tabular}{|c|p{5.5cm}|c|c|}
		\hline
		Computer & \centering{Prozessor} & RAM & Betriebssystem \\	
		\hline
		PC 1 &Intel(R) Core(TM) i7-4500 CPU @ 1.80GHz x 4 & 8,00 GB & Ubuntu 18.04.1 LTS\\
		PC 2 & Intel(R) Core(TM) i5-2430 CPU @ 2.40GHz & 8,00 GB & Windows 10 Home Premium\\
		PC 3 & 4 x Intel(R) Xeon(R) CPU @ 2.93GHz  & 12,30 GB & \\%Linux linux 3.13.0-162-generic #212-Ubuntu SMP Mon Oct 29 12:08:50 UTC 2018 x86_64 x_86_64 x86_64 GNU/Linux\\
		\hline 
	\end{tabular}
	\caption{Technische Daten} 
\end{table}
\newpage

\section{Vorüberlegung}
\label{preprocess}

Um jeden Kunden seinem Bedarf entsprechend zu versorgen, muss die Nachfrage jedes Kunden gedeckt sein.
Diese Bedingungen k\"onnen wir mit einfachen Vor\"uberlegungen decken.

Da die Kapazität der Glasfaserleitungen in unserem Modell unendlich ist, wird die Nachfrage eines Kunden auf jeden Fall gedeckt, wenn dieser mit Glasfaser angebunden ist.
Außerdem kann auf den inneren Kanten nur Glasfaser verlegt werden.
Das heißt, die einzigen Kanten, die dafür sorgen könnten, dass die Nachfrage eines Kunden nicht gedeckt ist, sind die Anschlusskanten von Typ 2.
Die Anschlusskanten von Typ 2 gehen von einer Facility zu einem Kunden.
Falls die Kapazit\"at dieser Kante nicht mindestens der Nachfrage des Kunden entspricht, können wir die Kante in unserem Netzwerk nicht benutzen. 
Deswegen löschen wir, bevor wir die verschiedenen Probleme lösen, diese Kanten aus dem oben genannten Graphen.
Das heißt jedes Netzwerk in dem neuen Graphen erfüllt somit die Nachfrage aller Kunden.
Daher betrachten wir die Nachfrage der Kunden ab hier nicht weiter.
Diesen neuen Graphen bezeichnen wir im Folgenden wieder mit $G=(V,A)$ und lösen die folgenden Probleme auf diesem Graphen.
Da die gelöschten Kanten in keiner der Lösungen benutzt werden könnten, verändert dies die Lösungen nicht.

\section{Point-to-point}

Wir behandeln im Folgenden die P2P Problemstellung.
Zuerst betrachten wir den Fall, dass s\"amtliche Kunden \"uber Glasfaserkanten angeschlossen werden m\"ussen.
Danach analysieren wir die Variante, dass sowohl Kupfer- als auch Glasfaserverbindungen erlaubt sind und zum Schluss dieses Kapitels ber\"ucksichtigen wir die M\"oglichkeit, dass die Kunden \"uber verschiedene Anschlusskanten auch unterschiedliche monatliche Profite f\"ur das Telekommunikationsunternehmen generieren oder die Nachfrage nach Bandbreite sich erh\"oht.

\subsection{Point-to-point mit Glasfaser}

Beim P2P mit Glasfaser Problem (P2PG) muss jeder Kunde mit einer eigenen Glasfaserleitung angeschlossen werden.
Das heißt, unser Problem besteht darin, eine Leitstelle aus $L$ auszusuchen und dann ein Glasfaserkabel von dieser Leitstelle zu einem Facility Knoten von Typ 1 zu verlegen, um dann mit der Anschlusskante von der Facility zum Kunden den Kunden anzubinden.
Dabei sollen die Kosten minimiert werden.
Es gibt genau eine Anschlusskante und eine Facility von der ein Kunde mit Glasfaser angebunden werden kann und die Länge dieser Kante ist 0.
Damit kann man das Problem einen Kunden anzuschließen, auch als kürzestes Weg Problem von einer Leitstelle zu der Facility des Kunden betrachtet werden.
Wir lösen also für jede Leitstelle in $L$ einmal das kürzeste Wege Problem zu jeder Facility von Typ 1.

Dabei gehen wir wie folgt vor:
Für jede Leitstelle $ l \in L$ berechnen wir auf dem ungerichteten Hilfsgraphen $H=(\{l\} \cup S \cup F , I,c\mid_I)$ einen kostenminimalen Weg von der Leitstelle $l$ zu jeder Facility $f \in F_1$. Dazu führen wir den Algorithmus von Dijkstra zur Berechnung k\"urzester Wege einmal mit Startknoten $l$ durch.
Der errechnete Weg zur Facility \( f \in F_1\) sei nun $W_{l,f}$ und die Kosten dieses Weges seien $c_{\text{dij}}(lf)$. 
Jeder Kunde besitzt nur eine eingehende Anschlusskante von Typ 1, diese hat Verlegungskosten 0. 
Außerdem besitzt jeder Kunde genau eine Facility aus $F_1$ von der dieser Kunde mit Glasfaser angeschlossen werden kann. 
Andersherum besitzt jede Facility aus $F_1$ nur einen Kunden, den diese anschließen muss. 
Daher sind die Kosten der optimalen Lösung für die ausgewählte Leitstelle $l$ nun $\text{c}_l:=c(l) + \displaystyle\sum_{f \in F_1} c_{\text{dij}}(lf) + c(f)$. 
%Also die Kosten die Wege von der Leitstelle $l$ zur jeder Facility aus $F_1$ plus die Kosten für die Anschlüsse aller Kunden an die Facility aus $F_1$.
%\textcolor{red}{TODO: sollte das l in L unter argmin nicht zentral sein?}
Nun suchen wir die Leitstelle $i \in L$ mit den geringsten Kosten ($i:=\arg \displaystyle\min_{l \in L} \text{c}_l$). Das zugehörige Netzwerk setzt sich aus den errechneten kürzesten Wegen $W_{i,f}$ für alle $f \in F_1$ und allen Anschlusskanten von Typ 2 zusammen, also ist das Netzwerk die Menge\textcolor{red}{TODO BESPRECHEN DAS IST IA KEINE DISJUNKTE VERINIGUNG} $(\bigcup_{f \in F_1 }W_{i,f}) \cup A_2 $. Die disjunkte Vereinigung bedeutet, dass falls zwei Wege über dieselbe Kante laufen, wir in unserem Netzwerk auch zwei Glasfaserkabel über diese Kante verlegen.

In diesem Algorithmus wenden wir also $|L|$ Mal den Dijkstra-Algorithmus an.
Dieser hat eine polynomielle Laufzeit von \TODO??. 

\TODO??
\textbf{Ergebnisse:} Die errechneten Kosten und die Laufzeiten sind der in folgenden Tabelle 2 dargestellt.
\begin{table}[h]
	\centering
	\begin{tabular}{c|c|c|c}
		 Instanz & Naunyn & Berlin & Vehlefanz \\	
		\hline
		Kosten & 480.722 & 1.388.562 & 7.791.258 \\
		Laufzeit auf PC2 & 0,001s & 0,26s & 2,35s\\
	\end{tabular}
	\label{P2PG}
	\caption{Ergebnisse des P2PG} 
\end{table}
Im Anhang befinden sich graphische Darstellungen zu diesen Instanzen.



\subsection{Point-to-point mit Glasfaser und Kupfer}
Das Point-to-point mit Glasfaser und Kupfer Problem (P2PGK) ist eine Erweiterung des P2PG Problems.
Hierbei d\"urfen Kunden nun auch \"uber Kupferkabel angeschlossen werden.
Dabei k\"onnen diese Kupferkabel lediglich als letzte Verbindungskanten von einer Facility von Typ 2 zu einem Kunden verlegt werden.
Ferner muss, falls ein Kunde von einer Facility aus mit Kupfer angeschlossen wird, in dieser Facility ein DSLAM Multiplexer installiert werden, der Kosten verursacht.
Jedoch k\"onnen von einer Facility mit installiertem DSLAM beliebig viele Kupferkabel verlegt werden und es muss lediglich eine Glasfaserverbindung in diese Facility daf\"ur eingehen.

Für jede Leitstelle $l \in L$ führen wir den folgenden Algorithmus durch.
Zuerst berechnen wir für alle Facility Knoten $f \in F$ einen kostenminimalen Weg von der Leitstelle $l$ zu der Facility $f$, genau wie beim P2PG Problem. Dazu führen wir einmal den Algorithmus von Dijkstra für das kürzeste Wege Problem auf dem Hilfsgraphen $H=(\{l\} \cup S \cup F , I,c\mid_I)$ mit Startknoten $l$ aus. Der errechnete Weg sei nun $W_{l,f}$ und die Kosten dieses Weges seien $c_{\text{dij}}(lf)$.
Für unsere Lösung heißt das, falls wir einen Kunden über die Facility $f$ anschließen, verlegen wir das Glasfaserkabel von $l$ nach $f$ genau auf dem kostenminimalen Weg $W_{l,f}$.  Jetzt müssen wir nur noch entscheiden, welchen Kunden wir an welche Facility anschließen.
Dazu konstruieren wir einen weiteren Hilfsgraphen. Sei dazu $E=\{lf \mid f \in F  \}$ die Menge der Kanten von der auswählten Leitstelle zu jeder Facility. Diese Kanten $lf$ ersetzen den errechneten kürzesten Weg $W_{l,f}$ von $l$ nach $f$. Der Hilfsgraph $H'=(V',A')$ besteht nun aus den Knoten $V'=\{l\} \cup F \cup K$ und den Kanten $E$ zusammen mit den Anschlusskanten von Typ 1 und Typ 2 ($A'=E \cup A_1 \cup A_2$), wie in Abbildung \ref{H''} dargestellt.

\begin{figure}[h]
\begin{tikzpicture}[->,>={Stealth[round,sep]},shorten >=1pt]
\node[shape=circle,draw=black] (1) at (0,0) {$l$};
\node[shape=circle,draw=black] (2) at (-4,-2) {$f_1$};
\node[shape=circle,draw=black] (3) at (-2,-2) {$f_2$};
\node[shape=circle,draw=black] (4) at (2,-2) {$f_{n-1}$};
\node[shape=circle,draw=black] (5) at (4,-2) {$f_n$};
\node[shape=circle,draw=black] (6) at (-5,-4) {$k_1$};
\node[shape=circle,draw=black] (7) at (-3,-4) {$k_2$};
\node[shape=circle,draw=black] (8) at (-1,-4) {$k_3$};
\node[shape=circle,draw=black] (9) at (3,-4) {$k_{m-1}$};
\node[shape=circle,draw=black] (10) at (5,-4) {$k_m$};
%\node at (0,-1) {\ldots};
\node at (0,-2) {\ldots};
%\node at (0,-3) {\ldots};
\node at (1,-4) {\ldots};
\node at (-6.5,0) {Leitstelle};
\node at (-6.5,-2) {Facilitys};
\node at (-6.5,-4) {Kunden};
\path (1) edge node[left, pos = 0.5] {} (2);
\path (1) edge node[left, pos = 0.5] {} (3);
\path (1) edge node[right, pos = 0.6] {} (4);
\path (1) edge node[right, pos = 0.5] {} (5);
\path (2) edge node[right, pos = 0.5] {} (6);
\path (2) edge node[right, pos = 0.5] {} (7);
\path (3) edge node[right, pos = 0.5] {} (6);
\path (3) edge node[right, pos = 0.5] {} (7);
\path (3) edge node[right, pos = 0.5] {} (8);
\path (4) edge node[right, pos = 0.5] {} (9);
\path (5) edge node[right, pos = 0.5] {} (10);
\end{tikzpicture}
\caption{Hilfsgraph $H'$} \label{H''}
\end{figure}
Außerdem definieren wir eine Kostenfunktion $c'$ auf den Kanten des Hilfsgraphen wie folgt:
\begin{align*}
c': A' \rightarrow \Q_{\geq 0}, ij \mapsto \left\{\begin{array}{cl} 
c_{\text{dij}}(lj), & \text{falls } ij \in E \text{ und } j \in F_1\\ 
c_{\text{dij}}(lj)+c(j), & \text{falls } ij \in E \text{ und } j \in F_2\\ 
c(ij) + c(i), & \text{falls } ij \in A_1\\ 
c(ij), & \text{falls } ij \in A_2\\ 
\end{array}
\right.
\end{align*}
Für die Kanten $lf$ ergeben sich Kosten von $c_{\text{dij}}(lf)$ für die Verlegung von Glasfaser auf dem Weg von $l$ nach $f$ für alle $f \in F$. Falls $f \in F_2$ eine Facility von Typ 2 ist, addieren wir noch Kosten von $c(f)$ auf die Kante, da wir einen DSLAM auf dieser Facility installieren m\"ussen, falls wir diese Kante sp\"ater benutzen. Für die Kanten $ij \in A$ gibt es Kantenkosten von $c(ij)$. Außerdem addieren wir für $ij \in A_1$ noch die Kosten für den Glasfaseranschluss $c(i)$ auf diese Kante, da $c(ij)=0$ in unseren Instanzen ist und wir nur eine Kante von der Facility $i$ haben, ist es egal, ob wir diese auf die Kanten auf $E$ addieren oder auf die Kanten aus $A_2$.

Dann lösen wir das Steinerbaum Modell mit der Fluss-Formulierung wie in der Vorlesung beschrieben\textcolor{red}{Nicht auf die Vorlesung verweisen MODEL EINFÜGEN}. Dabei wählen wir den Graphen $H'=(V',A')$ mit Kantenkosten $c'$, die Terminals wählen wir als die Kunden $K \cup \{l\}$ und $l$ w\"ahlen wir als Wurzel. Die Lösung des Steinerbaum Modells ist nun ein Baum $T_l$ mit Wurzel $l$, der jeden Kunden mit der Wurzel verbindet. Also ist jeder Kunde mit der Leitstelle $l$ über 2 Kanten verbunden. Die erste Kante ist eine Kante aus $E$ und die zweite Kante ist eine Anschlusskante aus $A_1$ oder $A_2$. Für jede Leitstelle werden mit diesem Modell Kosten $c'(T_l)$ berechnet. Addiert man auf diese Kosten noch die Kosten für den Aufbau der Leitstelle $l$ hinzu, erhält man die Kosten der Lösung für die Leitstelle \(l\), $\text{c}_l:=c(l)+c'(T_l)$.

Um nun eine Lösung des Problems zu erhalten, suchen wir die Leitstelle $i \in L$ mit den geringsten Kosten, $i:=\arg \displaystyle\min_{l \in L} \text{c}_l$. Das zugehörige Netzwerk setzt sich aus den errechneten kürzesten Wegen  $W_{i,f}$ und dem Steinerbaum $T_i$ zusammen. Für jede Kante $if \in T_i$ fügen wir den Weg $W_{i,f}$ zu unserem Netzwerk hinzu. Darüber hinaus fügen wir jede Anschlusskante aus $T_i$ hinzu. Also ergibt sich insgesamt das Netzwerk als die Menge der \textcolor{red}{WIEDER KEINE DISJUNKTEN WEGE IA} Kanten $(\dot{\bigcup}_{if \in T_i \cap E'}W_{i,f}) \cup (T_i\cap A)$.
\textcolor{red}{finde den n\"achsten satz nicht sch\"on}
Nochmal zur Zusammenfassung ist hier der Algorithmus aufgeschrieben.

\vspace{0.5cm}
\begin{algorithm}[H]
	\label{alg1}
	\SetKwInOut{Input}{Eingabe}\SetKwInOut{Output}{Ausgabe}
	\Input{$G=(V,A,c)$, wie in Kapitel 1 beschrieben}
	\Output{Leitstelle,Netzwerk,Kosten}
\BlankLine

\ForAll{$l\in L$}{
	\ForAll{$f \in F$}{
	Berechne $W_{l,f}$ und $c_{\text{dij}}(\{l,f\})$ mit dem Dijkstra-Algorithmus auf $H$}
	Berechne Steinerbaum $T_l$ auf $H'$ mit Wurzel $l$, Terminals $K \cup \{l\}$ und Kosten $c'$\\
	$\text{Kosten}_l:=c(l)+c'(T_l)$ \\
	}
	Leitstelle$:=i:=\arg \displaystyle\min_{l \in L} \text{Kosten}_l$\\
	Netzwerk$:=(\dot{\bigcup}_{if \in T_i \cap E'}W_{i,f}) \cup (T_i\cap A)$\\
	Kosten$:=\text{Kosten}_{i}$
	\BlankLine
\caption{Algorithmus zum Lösen des P2PGK Problems}
\end{algorithm}
\vspace{0.5cm}
Jetzt bleibt nur noch zu zeigen, dass der Algorithmus das optimale Ergebnis des P2PGK Problems liefert.
\begin{satz}
	Die Lösung des Algorithmus \ref{alg1} ist eine optimale Lösung für das P2PGK Problem.
\end{satz}
\begin{proof}
	\TODO
\end{proof}
\TODO LAUFZEIT EIN WENIG ANALYSIEREN???

\TODO Ergebnisse + Bilder als Übersicht:
\textbf{Ergebnisse:} 
\begin{table}[h]
	\centering
	\begin{tabular}{c|c|c|c}
		Instanz & Naunyn & Berlin & Vehlefanz \\	
		\hline
		Kosten & 397.662 & 524.392 & 647.608 \\
		Laufzeit auf PC2 & 0,03s & 2,89s & 144,61s \\
	\end{tabular}
	\label{P2PGK}
	\caption{Ergebnisse des P2PGK} 
\end{table}

\subsection{Erweiterung des P2PGK}
Wir betrachten nun zwei weitere Parameter des P2PGK.
Einerseits k\"onnen Kunden je nach Anschlusstechnologie monatliche Profite f\"ur das Telekommunikationsunternehmen verursachen und andererseits kann der Bedarf der Kunden in der n\"achsten Zeit steigen.

Um die steigende Nachfrage an Bandbreite der Kunden zu modellieren, haben wir den angegebenen Bedarf mit verschiedenen Werten $d$ multipliziert und dann, wie im Abschnitt \ref{preprocess}, die Kupfer Anschlusskanten gelöscht, die den neuen Bedarf der Kunden nicht decken können.
Dann lösen wir das P2PGK Problem erneut. 

\textbf{Auswertung:}
Betrachten wir zuerst die kleinste Instanz, Naunyn. Dazu erhöhen wir in $0,5$-Schritten den Bedarfsfaktor $d$. Bei $d=1$, also bei dem normalen P2PGK Problem, wird jeder Kunde mit Kupfer angeschlossen. Erhöht man den Bedarf um die Hälfte des vorhandnen Bedarfs, also $d=1,5$, müssen zwei Kunden mit Glasfaser angeschlossen werden. Dabei erhöhen sich natürlich auch die Kosten von vorher $397.662$ auf $442.842$.
\TODO? Jedoch ist die Entwicklung, dass sich der Bedarf um die Hälfte erhöht, sehr wahrscheinlich und somit ist es ratsam direkt diese Kunden mit Glasfaser anzuschließen und die höheren Kosten zu investieren.
Ab einem Bedarfsfaktor von $d=3,5$ werden 3 Kunden an Glasfaser angeschlossen. Erst bei einem Faktor von $d=6,5$ wird auch der letzte Kunde mit Glasfaser angeschlossen. 
\TODO? Es ist also kostensparender diesen Kunden erstmal mit Kupfer anzuschließen. Die Abbildung \ref{P2PGK_Naunyn_Bedarf} veranschaulicht diese Entwicklung nochmal.

\begin{figure}[h]
	\centering
	\begin{minipage}[b]{0.4\textwidth}
		\includegraphics[width=\textwidth]{./Bilder/P2PGK_Naunyn_demand1_duration0}
	\end{minipage}
	\begin{minipage}[b]{0.4\textwidth}
		\includegraphics[width=\textwidth]{./Bilder/P2PGK_Naunyn_demand1_5_duration0}
	\end{minipage}
	\begin{minipage}[b]{0.4\textwidth}
		\includegraphics[width=\textwidth]{./Bilder/P2PGK_Naunyn_demand3_5_duration0}
	\end{minipage}
	\begin{minipage}[b]{0.4\textwidth}
		\includegraphics[width=\textwidth]{./Bilder/P2PGK_Naunyn_demand6_5_duration0}
	\end{minipage}
	\caption{Lösung des P2PGK mit verschiedenen Bedarfsfaktoren $d$ auf der Instanz Naunyn. Links oben: $d=1$, rechts oben: $d=1,5$, links unten $d=3,5$, rechts unten: $d=6,5$}
	\label{P2PGK_Naunyn_Bedarf}
\end{figure}

Die gleiche Analyse für Berlin und Vehlefanz fassen wir nun in Tabelle \ref{P2PGK_Berlin_Bedarf} und \ref{P2PGK_Vehlefanz_Bedarf} zusammen. Dabei steht $A_k$ für die Anzahl der Kunden, die mit Glasfaser angeschlossen werden. In Berlin gibt es insgesamt 39 Kunden und in Vehlefanz 238 Kunden.

\begin{table}[h]
	\centering
	\begin{minipage}{.35\textwidth}
			\centering
	\begin{tabular}{c|c|c}
		\centering
		d& Kosten & $A_k$ \\	
		\hline
		1& 524.392 & 3 \\
		1,5 & 709.412 & 10 \\
		2 & 834.472 & 15 \\
		2,5 & 1.046.842 & 22 \\
		3 & 1.142.322 & 28 \\
		3,5 & 1.152.122 & 28 \\
		4 & 1.174.552 & 30 \\
		4,5 & 1.220.122 & 33 \\	
		5 & 1.220.122 & 33 \\
		$\infty$ &  1.388.562 & 39 \\
	\end{tabular}
	\label{P2PGK_Berlin_Bedarf}
	\caption{Lösung des P2PGK auf der Instanz Berlin mit veschiedenen Bedarfsfaktoren}
	\end{minipage}
	\hspace{0.5cm}
	\begin{minipage}{0.35\textwidth}
			\centering
		\begin{tabular}{c|c|c}
			\centering
			$d$ & Kosten & $A_k$ \\	
			\hline
			$1$   &   647.608 & 0  \\
			$1,5$ &   647.608 & 0  \\
			$2$   &   826.768 & 3  \\
			$2,5$ &   907.508 & 7  \\
			$3$   & 1.274.782 & 15 \\
			$3,5$ & 1.924.742 & 40 \\
			$4$   & 2.137.362 & 47 \\
			$4,5$ & 2.325.942 & 56 \\
			$5$   & 2.350.572 & 57 \\
			$\infty$ & 7.791.258 & 239 \\ 
		\end{tabular}
		\label{P2PGK_Vehlefanz_Bedarf}
	\caption{Lösung des P2PGK auf der Instanz Vehlefanz mit veschiedenen Bedarfsfaktoren $d$}
	\end{minipage}
\end{table}
\vspace{0.5cm}
In der Tabelle \ref{P2PGK_Berlin_Bedarf} erkennen wir, dass bei einem Bedarfsfaktor $d$ von nur $1,5$ schon 10 Kunden mit Glasfaser anschließen und die Zahl stetig steigt. Bei einem Wert $d=5$ werden schon 33 von 39 Kunden mit Glasfaser angeschlossen.
Bei der Instanz Vehlefanz hingegen werden bei $d=5$ erst 57 von 239 Kunden mit Glasfaser angeschlossen. Ingesamt steigt die Anzahl der mit Glasfaser angeschlossenen Kunden langsamer als bei der Instanz Berlin.
\TODO IST DAS GENUG DIE TABELLE AUSGEWERTET???

Nun erweitern wir das P2PGK Problem, indem wir zusätzlich noch den Profit für den Anschluss eines Kunden  betrachten.
Der Profit pro Monat für den Anschluss eines Kunden $k \in K$ beträgt für Glasfaser $p_1(k)$ und für Kupfer $p_2(k)$, im folgenden betrachten wir dies als Profitfunktion $p:K \times \{1,2\} \rightarrow \Q,(k,i) \mapsto p_i(k)$. Dazu führen wir die Variable $t$ ein. Für $t=0$ ist die Lösung die Lösung für das P2PGK, wie im Abschnitt zuvor. $t=1$ ist das erste Jahr mit Profit, und so weiter. Um diese Erweiterung unseres P2PGK Problems zu lösen, übergeben wir dem Algorithmus  \ref{alg1} hierbei noch zusätzlich $t$ und $p$ und passen die Kostenfunktion $c'$ wie folgt an.
\begin{align*}
c': A'' \rightarrow \Q, \{ i,j \} \mapsto \left\{\begin{array}{cl} 
c_{\text{dij}}(lf), & \text{falls } ij \in E \text{ und } j \in F_1\\ 
c_{\text{dij}}(lf)+c(j), & \text{falls } ij \in E \text{ und } j \in F_2\\ 
c(ij) + c(i) - t \cdot p_1(j), & \text{falls } ij \in A_1\\ 
c(ij) - t \cdot p_2(j), & \text{falls } ij \in A_2\\ 
\end{array}
\right.
\end{align*}
Außerdem erweitern wir, dass Steinerbaum Model mit der Bedingung 
\begin{align*}
	x_{ij} \leq \displaystyle\sum_{t \in K} y_{ij}^t \quad \forall ij \in A \text{ mit } c(ij) \leq 0
\end{align*}
Diese Nebenbedingungen ist notwendig, da sonst alle Kanten $ij$ mit negativen Kosten ausgewählt werden. Die Nebenbedingungen verhindert dies, da nun nur die Kanten ausgewählt werden können, welche auch wirklich in dem Fluss vorkommen.
Es ist leicht zu sehen, dass der Algorithmus \ref{alg1} jetzt auch hierfür die optimale Lösung ausgibt. 

\textbf{Auswertung:}
Zuerst stellen wir fest, dass der Profit einen Kunden $k \in K$ mit Glasfaser anzuschließen oft sich nur um 1/3 höher als der Profit p $p_2(k)$. 
Wir betrachten die Veränderung der Ergebnisse in 10-Jahres Schritten, da ein großer Planungshorizont bei diesem Projekt mehr Sinn macht.
Betrachten wir hier nun die Instanz Berlin. Für $t=0$ haben wir Kosten von 524.392 und es werden 3 Kunden mit Glasfaser angeschlossen. Nach 10 Jahren sind die Kosten gesunken auf 366.112 bei gleich bleibender Anzahl von Kunden mit Glasfaseranschluss. Es ist sehr wahrscheinlich, dass sich das Netzwerk hier nicht ändert. Erst bei einem Planungshorizont von 40 Jahren ändert sich das Netzwerk, aber es wird nur ein Kunde mehr mit Glasfaser angeschlossen. Da sich in 40 Jahren sehr viel ändern kann, ist es nicht sinnvoll den Profit mit in die Planung einzubeziehen. Für Naunyn und Vehlefanz ergibt die Analyse des Profit dasselbe. Also sollte das Netzwerk aufgebaut werden, dass auch herauskommt, wenn man ohne Profit rechnet. Die genauen 

Zusammenfassend ist es in unseren Instanzen beim P2P- Problem wichtiger eine steigende Nachfrage zu betrachten als den Profit für

\section{Point-to-multipoint communication}
\TODO EINLEITUNG UND PROBLEM BESCHREIBUNG

 
 \subsection{Point-to-mulitpoint mit Glasfaser}
\TODO PROBELM BESCHREIBUNG $a$ wie viele Kabel kann der splitter splitten

Dieses Problem lösen wir, indem wir das folgende Model für jede Leistelle $l \in L $ mit Gurobi lösen. Sei dazu $A:= I\backslash I_L \cup A_1$, wobei $I_L$ die Menge der ausgehenden Kanten aus den nicht ausgewählten Leitstellen ist.
Die Entscheidungsvariable $y_{ij}^t \in \{0,1\}\; \forall ij \in A, t \in K$ modelliert einen Fluss von der Leitstelle $l$ zu dem Kunden $t$ für jeden Kunden. Falls $y_{ij}^t=1$ ist, heißt das, dass der Fluss von $l$ nach $t$ über die Kante $ij$ fließt.
Die Variable $x_{ij} \in \Z \forall ij \in A $ gibt an, wie häufig die Kante in dem Netzwerk auftritt, bzw. wie häufig wir eine Kante kaufen müssen. Außerdem haben wir die binäre Entscheidungsvariable $s_i \in \{0,1\}$ für $i \in F \cup S$ eingeführt. Diese gibt an, ob in dem Knoten $i$ ein Splitter installiert wird ($s_i=1$) oder nicht ($s_i=0$). In dem Model wollen wir nun die Kosten für die Installation eines Splitter und die Kosten für die Kanten minimieren. Dabei haben wir die Kosten für den Glasfaseranschluss auf die Kantenkosten der Anschlusskanten für Glasfaser addiert.
\begin{align*}
c': A \rightarrow \Q, ij  \mapsto \left\{\begin{array}{cl} 
c(ij), & \text{falls } ij \in I\backslash I_L\\ 
c(ij)+c(j), & \text{falls } ij \in A_1\\ 
\end{array}
\right.
\end{align*}
Also ergibt sich die folgende Minimierungsfunktion und insgesamt das folgende Modell.


 \bigskip
 $\min \displaystyle\sum_{ij \in A} c'_{ij} x_{ij} + \displaystyle\sum_{i \in F \cup S} c_s s_i $
 \begin{align*}
 \begin{array}{rcrcrcll}
 \textrm{s.t.}  
&& &\displaystyle\sum_{ji \in A} y_{ji}^t - \displaystyle\sum_{ij \in A} y_{ij}^t& = & \left\{\begin{array}{cl} 
 -1, & \text{falls } i=l\\ 
 1, & \text{falls } i=t\\ 
 0, & \text{sonst.}\\ 
 \end{array}
 \right. & \forall t \in K & (1) \\
 &&& y_{ij}^t & \leq & x_{ij} & \forall ij \in A, t\in K & (2)\\
 &&& s_i &\leq& \displaystyle\sum_{ji \in A} x_{ji}& \forall  i \in S \cup F& (3)\\ 
 &0&\geq&\displaystyle\sum_{ji \in A} x_{ji} - \displaystyle\sum_{ij \in A} x_{ij}&\geq& -(a-1)s_i & \forall i \in S \cup F& (4)\\
 &&& y_{ij}^t & \in & \{0,1 \}& \forall ij \in A, t \in K & (5)\\
 &&& x_{ij} & \in & \Z & \forall ij \in A, t \in K & (6)\\
 &&& s_i & \in & \{ 0,1 \} & \forall i \in F \cup S & (7) \\
 \end{array}
 \end{align*}
 Die Nebenbedienung (1) modelliert, wie im Steinerbaum Problem mit Flussbedingung, einen Fluss für jeden Kunden $t \in K$ von der Leitstelle $l$ zu dem Kunden. Dabei gibt die Formel $\displaystyle\sum_{ji \in A} y_{ji}^t - \displaystyle\sum_{ij \in A} y_{ij}^t$ die Flusserhaltung an. Falls diese Formel gleich -1 eins ist, geht als ein Fluss aus dem Knoten heraus. Dabei geht keine Kante in die Leitstelle ein, da die Leitstelle nur ausgehende Kanten besitzt. Das gleiche gilt für die Kunden, diese haben nur eingehende Kanten und somit gilt, falls die Formel gleich 1 ist, dass genau ein Flusskante in den Kundenknoten geht. Die zweite Nebenbedingung stellt sicher, dass falls wir eine Kante in dem Fluss von $l$ nach $t$ benutzen, wird diese Kante auch in unserem Netzwerk gekauft. Falls wir einen Splitter auf einem Knoten $i \in S \cup F$ benutzen, braucht dieser Splitter auch ein Kabel, dass er splitten kann. Dies gewährleistet Bedingung (3). Außerdem besitzen die $x_{ij}$-Variablen auch Flusserhaltung, modelliert in Bedingung (4). Dies ist notwendig, da wir nur dann die Kabel aufteilen können, wenn wir auch einen Splitter installieren. Falls wir keinen Splitter installieren, also $s_i=0$, müssen genauso viele Kabel in den Knoten rein gehen, wie auch wieder raus gehen, dh. $\displaystyle\sum_{ji \in A} x_{ji} - \displaystyle\sum_{ij \in A} x_{ij}=0$. Falls wir einen Splitter installieren ($s_i=1$), können bis zu $a$ Kanten aus diesem Knoten heraus gehen. Deshalb muss die Flusserhaltung der $x_{ij}$ größer als $-(a-1)$ sein. \TODO NEBENBEDINUNG(4)KACKE
 Die Nebenbedienungen (5)-(7) wurden über dem Model erklärt.
  
 \TODO wieso ist das die optimale lösung also wieso schränken wir keine vll besseren 
 lösungen ein
 \TODO Ergebnisse beschreiben +splitter 4 oder 16 oder noch andere splitter
 \begin{table}[h]
 	\centering
 	\begin{tabular}{c|c|c|c}
 		Instanz & Naunyn & Berlin & Vehlefanz \\	
 		\hline
 		Kosten für $a=4$ & \TODO &  &  \\
 		Kosten für $a=16$ & \TODO &  &  \\
 		Laufzeit auf PC2 &  &  & \\
 	\end{tabular}
 	\label{P2MPG}
 	\caption{Ergebnisse des P2MPG} 
 \end{table}
 \TODO vergleich von 4 und 16 
 
 \subsection{Point-to-mulitpoint mit Glasfaser und Kupfer}
\TODO PROBLEM beschreibung+einleitung

Diese Problem lösen wir mit einer Erweiterung des linearen Programms des P2MPG Problems. Seinen also die Entscheidungsvariablen $y_{ij}^t,x_{ij}$ und $s_i$ definiert wie in dem P2MPG Model. Wir fügen noch eine Entscheidungsvariable $m_i \in \{0,1\} \forall i \in F_2$. In jeder Facility von Typ 2 kann bei diesem Problem noch ein DSL-Zugangsmultiplexer installiert werden. Genau dann wenn $m_i=1$ ist, wird ein Multiplexer installiert.

  \bigskip
  $\min \displaystyle\sum_{ij \in A} c'_{ij} x_{ij} + \displaystyle\sum_{i \in F \cup S} c_s s_i + \displaystyle\sum_{i \in F_2} c(i) m_i$
  \begin{align*}
  \begin{array}{rcrcrcll}
  \textrm{s.t.}  
  && &\displaystyle\sum_{ji \in A} y_{ji}^t - \displaystyle\sum_{ij \in A} y_{ij}^t& = & \left\{\begin{array}{cl} 
  -1, & \text{falls } i=r\\ 
  1, & \text{falls } i=t\\ 
  0, & \text{sonst.}\\ 
  \end{array}
  \right. & \forall t \in K & (1) \\
  &&& y_{ij}^t & \leq & x_{ij} & \forall ij \in A, t\in K & (2)\\
    &&& s_i &\leq& \displaystyle\sum_{ji \in A} x_{ji}& \forall  i \in F \cup S & (3)\\ 
  &0&\geq&\displaystyle\sum_{ji \in A} x_{ji} - \displaystyle\sum_{ij \in A} x_{ij}&\geq& -(a-1)s_i & \forall i \in S \cup F_1& (4)\\
   &0&\geq&\displaystyle\sum_{ji \in A} x_{ji} -m_i - \displaystyle\sum_{ij \in A} x_{ij}&\geq& -(a-1)s_i & \forall i \in F_2& (4')\\
   &&&s_i+m_i & \leq & 1 & \forall i \in F_2 & (8)\\
   &&&x_{ij}& \leq & m_i & \forall i \in F_2 & (9) \\
    &&& y_{ij}^t & \in & \{0,1 \}& \forall ij \in A, t \in K & (5)\\
    &&& x_{ij} & \in & \Z & \forall ij \in A, t \in K & (6)\\
    &&& s_i & \in & \{ 0,1 \} & \forall i \in F \cup S & (7) \\
    &&& m_i & \in & \{ 0,1 \} & \forall i \in F_2 & (10) \\
  \end{array}
  \end{align*}
  
  Die Nebenbedingungen (4) aus dem P2MPG Model bleibt für die Knoten $S \cup F$ die gleiche. In den Knoten aus $F_2$ können wir nun auch eine Multiplexer installieren. Dies müssen wir auch bei der Flusserhaltung der $x_{ij}$ beachten. Die Nebenbedienungen (4') modelliert den Fluss der $x_{ij}$ für die Knoten $i \in F_2$. Falls ein Splitter installiert wird, kann kein Multiplexer installiert werden und die Bedienungen ist wie im P2MPG Model. Falls ein Multiplexer $i \in F_2$
   
   \TODO+wieso ist das die optimale lösung also wieso schränken wir keine vll besseren lösungen ein
   \TODO Ergebnisse beschreiben
   \begin{table}[h]
   	\centering
   	\begin{tabular}{c|c|c|c}
   		Instanz & Naunyn & Berlin & Vehlefanz \\	
   		\hline
   		Kosten für $a=4$ & \TODO &  &  \\
   		Kosten für $a=16$ & \TODO &  &  \\
   		Laufzeit auf PC2  &  &  & \\
   	\end{tabular}
   	\label{P2MPGK}
   	\caption{Ergebnisse des P2MPGK} 
   \end{table}
   
   
 \subsection{Erweiterung des P2MPGK}
 
 \section{Zusammenfassung/Fazit}
 \TODO Graphische verschauen der verschiedenen Ergebnisse der verschiedenen Probleme
 
 
\newpage
\bibliographystyle{babplain-lf}
\renewcommand{\refname}{Literaturverzeichnis}
\bibliography{literatureProject1}
\thispagestyle{empty}
\newpage
\appendix
\section*{Abbildungsverzeichnis}
\textbf{P2PG:}
\begin{figure}[h]
	\begin{center}
		\begin{minipage}{8.0cm}
			\includegraphics[width=1\textwidth]{./Bilder/P2PG_Naunyn}
			\caption{Lösung des P2PG Problems auf der Instanz Naunyn}
		\end{minipage}
	\end{center}
\end{figure}

\begin{figure}[h]
\begin{center}
	\begin{minipage}{12.0cm}
		\includegraphics[width=1\textwidth]{./Bilder/P2PG_Berlin}
		\caption{Lösung des P2PG Problems auf der Instanz Berlin}
	\end{minipage}
\end{center}
\end{figure}

\begin{figure}
\begin{center}
	\begin{minipage}{12.0cm}
		\includegraphics[width=1\textwidth]{./Bilder/P2PG_Vehlefanz}
		\caption{Lösung des P2PG Problems auf der Instanz Vehlefanz}
	\end{minipage}
\end{center}
\end{figure}

\addcontentsline{toc}{section}{Abbildungsverzeichnis}
\section*{Anhang}

\subsection*{}


\addcontentsline{toc}{section}{Anhang}
\thispagestyle{empty}


\end{document}
