\documentclass{beamer}
\usefonttheme[onlymath]{serif}

\mode<presentation> {
	\usetheme{Madrid}}
\usepackage{graphicx}
\usepackage{ucs}
\usepackage[utf8x]{inputenc}
\usepackage[T1]{fontenc}
\usepackage[ngerman]{babel}
\usepackage[ngerman]{varioref}
\usepackage{amsmath,amssymb,amstext,amsthm}
\usepackage{babelbib}
\usepackage[linesnumbered]{algorithm2e}

\newcommand{\R}{\mathbb{R}}
\newcommand{\N}{\mathbb{N}}
\newcommand{\Z}{\mathbb{Z}}
\newcommand{\Prim}{\mathbb{P}}
%\newcommand{\cha}{\text{$\unlhd$\raisebox{0pt}{$\stackrel{{}_\vert}{}$} }}
%\newcommand{\AutG}{\text{Aut($G$)}}
%\newcommand{\SocHi}{\text{\rm Soc($H_i$)}}
%\newcommand{\bigslant}[2]{\left.\raisebox{.2em}{$#1$}\middle/\raisebox{-.2em}{$#2$}\right.}

%\newcounter{saveenumi}
%\newcommand{\seti}{\setcounter{saveenumi}{\value{enumi}}}
%\newcommand{\conti}{\setcounter{enumi}{\value{saveenumi}}}


\title[Projekt 2]{Fiber-To-The-x}

\author{Lucia Ortjohann und Moritz Hefner}

\institute[]{Netzwerkoptimierung in der Praxis}

\date{16. November 2018}

\begin{document}
	\begin{frame} %C
		\titlepage
	\end{frame}
	
	\begin{frame}{Vor\"uberlegung}
	\begin{itemize}
		\item Jeder Kunde \(k \in K\) hat  eine Nachfrage an Bandbreite von $d(k)$.
		\item Glasfaserkabel liefern unendlich Bandbreite.
		\item Kupferkabel haben eine beschr\"ankte Kapazit\"at, also die Kanten aus $A_2$.
		\vspace{0.8cm}
		\pause
		\item F\"ur alle Kanten aus $ij \in A_2$ \"uberpr\"ufen wir:\\
		Falls die Kapazit\"at der Kante $ij$ kleiner ist als der Bedarf  $d(j)$,
		l\"oschen wir die Kante aus dem Graphen.
		\item Alle Probleme l\"osen wir auf diesem neuen Graphen. 
	\end{itemize}
		
	\end{frame}
	\begin{frame}{Point-to-point mit Glasfaser}
	\scalebox{1}{	\begin{algorithm}[H]
			\label{algP2PG}
			\SetKwInOut{Input}{Eingabe}\SetKwInOut{Output}{Ausgabe}
			\Input{$G=(V,A,c)$}
			\Output{Leitstelle,Netzwerk,Kosten}
			\BlankLine
			
			\ForAll{$l\in L$}{
				\ForAll{$f \in F_1$}{
					Berechne $W_{l,f}$ und $c_{\text{dij}}(\{l,f\})$ mit dem Dijkstra-Algorithmus auf $H=(\{l\} \cup S \cup F , I,c\mid_I)$}
				$C_l:=c(l) + \displaystyle\sum_{f \in F_1} c_{\text{dij}}(lf) + c(f)$. 
			}
			Leitstelle$:=i:=\arg \displaystyle\min_{l \in L} C_l$\\
			Netzwerk$:=(\dot{\bigcup}_{f \in F_1 }W_{i,f}) \cup A_2 $. \\
			Kosten$:=C_{i}$
			\BlankLine
			\caption{Algorithmus zum Lösen des P2PG Problems}
	\end{algorithm}}

	\end{frame}
	\begin{frame}{Point-to-point mit Glasfaser und Kupfer}
\scalebox{1}{	\begin{algorithm}[H]
		\label{alg1}
		\SetKwInOut{Input}{Eingabe}\SetKwInOut{Output}{Ausgabe}
		\Input{$G=(V,A,c)$}
		\Output{Leitstelle,Netzwerk,Kosten}
		\BlankLine
		
		\ForAll{$l\in L$}{
			\ForAll{$f \in F$}{
				Berechne $W_{l,f}$ und $c_{\text{dij}}(\{l,f\})$ mit dem Dijkstra-Algorithmus auf $H=(\{l\} \cup S \cup F , I,c\mid_I)$ mit Startknoten $l$}
			Berechne Steinerbaum $T_l$ auf $H'$ mit Wurzel $l$, Terminals $K \cup \{l\}$ und Kosten $c'$\\
			$C_l:=c(l)+c'(T_l)$ \\
		}
		Leitstelle$:=i:=\arg \displaystyle\min_{l \in L} \text{Kosten}_l$\\
		Netzwerk$:=(\dot{\bigcup}_{if \in T_i \cap E'}W_{i,f}) \cup (T_i\cap A)$\\
		Kosten$:=C_{i}$
		\BlankLine
		\caption{Algorithmus zum Lösen des P2PGK Problems}
	\end{algorithm}}
	

	\end{frame}
	\begin{frame}{Erweiterung des P2PGK}
	\textbf{Bedarf:}
	\textbf{Profit:}

	\end{frame}

	\begin{frame}{Point-to-multipoint mit Glasfaser}
	F\"ur jede Leistelle $l \in L$ l\"osen wir: \\
	Graph:  $G=(V:=\{l\}\cup S \cup F \cup K ,\: \:   A:= I\backslash I_L \cup A_1$)\\
	\vspace{0.2cm}
	 $\min \displaystyle\sum_{ij \in A} c'_{ij} x_{ij} + \displaystyle\sum_{i \in F \cup S} c_s s_i $
	\begin{align*}
\begin{array}{rcrcrcl}
\textrm{s.t.}  
&& &\displaystyle\sum_{ji \in A} y_{ji}^t - \displaystyle\sum_{ij \in A} y_{ij}^t& = & \left\{\begin{array}{cl} 
-1, & \text{falls } i=l\\ 
1, & \text{falls } i=t\\ 
0, & \text{sonst.}\\ 
\end{array}
\right. & \forall t \in K\\
&&& y_{ij}^t & \leq & x_{ij} & \forall ij \in A, t\in K \\
&&& s_i &\leq& \displaystyle\sum_{ji \in A} x_{ji}& \forall  i \in S \cup F\\ 
&0&\geq&\displaystyle\sum_{ji \in A} x_{ji} - \displaystyle\sum_{ij \in A} x_{ij}&\geq& -(a-1)s_i & \forall i \in S \cup F\\
&&& y_{ij}^t & \in & \{0,1 \}& \forall ij \in A, t \in K \\
&&& x_{ij} & \in & \Z_{\geq 0} & \forall ij \in A, t \in K \\
&&& s_i & \in & \{ 0,1 \} & \forall i \in F \cup S \\
\end{array}
\end{align*}
\end{frame}

\begin{frame}{Point-to-multipoint mit Glasfaser und Kupfer}
	F\"ur jede Leistelle $l \in L$ l\"osen wir: \\
    Graph:  $G=(V:=\{l\}\cup S \cup F \cup K ,\: \:   A:= I\backslash I_L \cup A$)\\


\end{frame}
\begin{frame}{Erweiterung des P2MPGK}

\end{frame}

\begin{frame}{Zusammenfassung}

\end{frame}

	
\end{document}